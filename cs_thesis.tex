\documentclass[a4paper, 12pt]{report}
\usepackage{cs_thesis}
\usepackage[pdftex]{graphicx}
\usepackage[utf8]{inputenc}
\usepackage{amsmath}
\usepackage{url}
\usepackage{biblatex}
\usepackage{listings}
\usepackage{color}

\renewcommand{\UrlFont}{\small\tt}

\lstset{language=C++,
  basicstyle=\ttfamily,
  keywordstyle=\color{blue}\ttfamily,
  stringstyle=\color{red}\ttfamily,
  commentstyle=\color{green}\ttfamily,
  morecomment=[l][\color{magenta}]{\#}
}

\lstset{language=C++,
  keywordstyle=\color{blue},
  stringstyle=\color{red},
  commentstyle=\color{green},
  morecomment=[l][\color{magenta}]{\#}
}

\title{The iCub Humanoid Robot}
\author{Semih Onay}
\program{Computer Science}
\supervisor{Associated Professor Elena Battini Sönmez}


\begin{document}
  \pagenumbering{roman}
  \makecstitle
  
  \tableofcontents
  \listoffigures
  \listoftables
  
  \begin{symabbreviations}
    \sym{\textbf{YARP}}{Yet Another Robot Platform}
    \sym{}{} 
    \sym{DA}{Description of abbreviation}
  \end{symabbreviations}
  
  {\tiny }\chapter*{ABSTRACT}
  Robot are evolving to the dead ends. Software and hardware they produce, 
  disappearing without trace afterwards. Here, exploring how to make our 
  projects stable and long-lasting, without yielding our ability to 
  constantly change our sensors, actuators, processors, and networks.
  We advance on two fronts, software and hardware. For some time, we have been 
  developing and using the YARP robot software architecture, which helps 
  organize communication between sensors, processors, and actuators so that 
  loose 
  coupling is encouraged, making gradual system evolution much easier. YARP 
  includes a model of communication that is transport-neutral, so that data 
  flow 
  is decoupled from the details of the underlying networks and protocols in 
  use. 
  Importantly for the long term, YARP is designed to play well with other 
  architectures.
  
  \begin{figure}[h!]
    \centering
    \includegraphics[width=0.6\linewidth]{sim}
    \caption{}
    \label{fig:sim}
  \end{figure}
  
  Device drivers written for YARP can be ripped out and used 
  without any “middleware.” On the network, basic interoperation is possible 
  with 
  a few lines of code in any language with a socket library, and maximally 
  efficient interoperation can be achieved by following documented protocols. 
  These features are not normally the first things that end-users look for when 
  starting a project, but they are crucial for longevity.
  We emphasize the strategic utility of the Free Software social contract to 
  soft- ware development for small communities with idiosyncratic requirements. 
  We also work to expand our community by releasing the design of our ICub 
  humanoid under a free and open license, and funding development using this 
  platform.
  
  \chapter*{ÖZET}
  
  \chapter{INTRODUCTION}
  
  The iCub is a humanoid robot developed at Istituto Italiano di Tecnologia 
  (IIT) 
  as part of the European project RobotCub and later adopted by more than 20 
  laboratories arround the world. It has 53 motors that can move the head, arms 
  and hands, midriff, and legs.It can see and hear, it has the sense of 
  proprioception (body 
  configuration) and movement (using accelerometers and gyroscopes).
  It’s designed to aid studies of human cognition and artificial 
  intelligence. Project members developed computer simulator to experiment new 
  techniques.
  \par Computer simulations are getting important in area robotics 
  area.Simulations may not provide real complexity of the physical world and 
  not 
  reliable as real dynamics. The simulator of iCub is an easy way to test new 
  algorithms and methods instead of dealing with complex configuration of iCub 
  hardware. The simulator is designed to be accurate as real world psychics and 
  dynamics. Development is based on directly from first prototype of simulation 
  environment \cite{Webots}. It was expensive and had limited access to 
  source code which made hard to modify source code in order to add some 
  functionalties.Then iCub simulation created. Simulation environment uses 
  \cite{ODE} to simulate body movement and collision detection algorithms to 
  measure psychical interaction with the world.ODE is used in wide range of 
  projects like \cite{Gazebo}. ODE is an open source physics engine for 
  authoring tools, computer 
  games,etc.It uses OpenGL renderer and it has some disadvantages due to 
  limitation of OpenGL engine computation efficiency on complex structures.iCub 
  simulation uses OpenGL directly via SDL which helps to render complex robot 
  movements and com- putation efficient simulation observations. Simulator is 
  free and available to anyone who interested in robotics and learning about 
  basics of robotics.
  
  
  \chapter{LITERATURE REVIEW}
  
  Development of simulator is described by abstraction of parts to handle 
  complex 
  instructions more precisely and efficient. Some other external software 
  libraries are used to reduce required time to animate given parts of robots 
  body from parameters. Abstractions made it easy to implement new 
  methods,algorithms into a simulation environment. Understanding of these 
  libraries are the key of creating new interfaces and methods to robot in 
  virtual world.
  Action primitives are pre-defined inside a simulation environment to extend 
  ca- 
  pability of creating new interfaces to virtual simulation world and can be 
  changed or taught as different languages which helps to extend knowledge 
  about 
  languages.
  
  \chapter{METHODOLOGY}
  
  \subsection{The YARP Arcitechture}
  The computer simulation model of the iCub robot.The simulator allows to 
  create realistic scenarios in where robot can interact with a virtual world 
  and 
  physical limitations and interactions that occur between the virtual world is 
  simulated using open source library which is \cite{ODE} (Open Dynamics 
  Engine) 
  to 
  provide accurate simulation of body dynamics.
  
  \begin{figure}[h!]
    \centering
    \includegraphics[width=1.0\linewidth]{cognitive_architecture}
    \caption{Cognitive Architecture of iCub}
    \label{fig:cognitive_architecture}
  \end{figure}
  
  Top of \cite{YARP}. It is a set of open source 
  libraries that supports modularity by using abstraction method in softwares 
  to 
  handle common difficulties in robotics area which are know as modularity 
  algorithms and hardware in- terfaces and OS platforms.To deal with OS 
  spesific 
  builds,requires to use cross-platform build tools such as \cite{Cmake} and 
  \cite{ACE}.
  YARP is providing platform independence.First abstraction can be described as 
  a 
  protocols.Main YARP protocol manages inter-process communications in 
  operating 
  systems.It can deliver process messages of any size across the network by 
  using 
  different protocols.
  
  Second abstraction is about hardware communications.The method is to define 
  interface for class of devices to fold native coded APIs.Changes in hardwares 
  requires changes in API calls via linking suitable libraries to encapsulate 
  hardware dependency problems. These two abstraction combined to use remote 
  device drives where that can be accessed across the network like a parallel 
  processing.
  
  The purpose of YARP ports are to move data from threads to threads over the 
  processes.Flow of the data can be configured and observed from command-line 
  at 
  real- time.Port can receive or send data from any other port.Connections 
  between ports can be modified easily with using different protocols such as 
  TCP(Transmission Control Protocol) and UDP(User Datagram Protocol).The choice 
  of protocol is depends on quality of message transmission or response 
  time.Using TCP is for reliability and UDP is for speed with effect on 
  unreliable transmissions.
  \begin{figure}[h!]
    \centering
    \includegraphics[width=0.8\linewidth]{yarp}
    \caption{}
    \label{fig:yarp}
  \end{figure}
  \newpage
  \subsection{Mechanics of iCub}
  
  2.1 Mechanics
  The kinematic specifications of the body of the iCub including the definition 
  of the number of DOF and their actual locations as well as the actual size of 
  the limbs and torso were based on ergonomic data and x-ray images.
  The possibility of achieving certain motor tasks is favored by a suitable 
  kinematics and, in particular, this translates into the determination of the 
  range of movement and the number of controllable joints (where clearly 
  replicating the human body in detail is impossible with current technology). 
  Kinematics is also influenced by the overall size of the robot which was 
  imposed a priori. The size is that of a 3.5 years old child (approximately 
  100cm high). This size can be achieved with current technology. QRIO1 is an 
  example of a robot of an even smaller size although with less degrees of 
  freedom. In particular, our task specifications, especially manipulation, 
  require at least the same
  
  
  \begin{figure}[h!]
    \centering
    \includegraphics[width=0.9\linewidth]{cognitive_architecture_A}
    \caption{}
    \label{fig:cognitive_architecture_A}
  \end{figure}
  
  
  \chapter{RESULTS}
  
  Results will be here ...
  
  \appendix
  \chapter{Sample Code Snippet}
  
  
  \begin{lstlisting}
  
  public:objectMoverThread(ResourceFinder &_rf) : rf(_rf) {
  virtual bool loadParams() {
  
  name = rf.check("name",Value("objectMover")).asString().c_str();
  neckTT = rf.check("necktt",Value(2.0)).asDouble();
  eyeTT = rf.check("eyett",Value(1.2)).asDouble();
  trajTime = rf.check("trajtime",Value(4.0),"Solver trajectory 
  time").asDouble();
  maxPitch = rf.check("maxpitch",Value(30.0),"Torso max pitch").asDouble();
  
  //get which arm to use. default to left if they didnt pass in left or right
  armname = rf.check("arm", Value("left"),"arm name").asString().c_str();
  if (armname == "right") {
  armInUse = true;
  }
  else {
  armInUse = false;
  }
  
  //get which robot target to use
  robotname = rf.check("robot", Value("icub"),"robot name").asString().c_str();
  
  }
  \end{lstlisting}
  
  \appendix
  \chapter{Screenshots}
  
  \begin{thebibliography}{9}
    \bibitem{Webots} 
    Webots: robot simulation software
    \\\texttt{https://www.cyberbotics.com}
    
    \bibitem{ODE} 
    ODE: Open Dynamics Engine
    \\\texttt{http://www.ode.org}
    
    \bibitem{Gazebo}
    Gazebo : Open Source Simulation Environment
    \\\texttt{http://gazebosim.org}
    
    \bibitem{YARP} 
    YARP: Yet Another Robot Platform 
    \\\texttt{http://wiki.icub.org/yarp/}
    
    \bibitem{CMake} 
    CMake: Cross-Platform Open Source Build System 
    \\\texttt{http://www.cmake.org}
    
    \bibitem{ACE} 
    ACE: The ADAPTIVE Communication Environment
    \\\texttt{http://www.cs.wustl.edu/~schmidt/ACE.html}
    
    \bibitem{iCub} 
    iCub: An Open Source Cognitive Humanoid Robotic Platform
    \\\texttt{http://www.icub.org}
    
  \end{thebibliography}
  
\end{document}